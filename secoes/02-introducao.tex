\section{Introdução}
A Inteligência Artificial consolidou-se como uma das tecnologias mais transformadoras da era contemporânea, impactando setores que vão desde a saúde e educação até a indústria e, de forma especialmente relevante, o desenvolvimento de software \cite{roberta2021}.  Entre os avanços mais significativos desse campo destacam-se os grandes modelos de linguagem, que revolucionaram a forma como sistemas computacionais interagem com linguagem natural e executam tarefas complexas.

Nos últimos anos, modelos como o ChatGPT da OpenAI passaram a desempenhar papel significativo no suporte ao estudo e à prática de desenvolvimento de \textit{software}, auxiliando na geração, revisão, documentação e depuração de código. Essa utilização oferece alternativas para a resolução de problemas e contribui para reduzir barreiras comuns enfrentadas por estudantes, como dificuldades de sintaxe, arquitetura de projetos ou compreensão de erros. Ao mesmo tempo, levanta discussões importantes sobre eficiência, confiabilidade e adequação desses modelos em diferentes contextos de ensino e desenvolvimento.

Dessa forma, a análise comparativa de desempenho torna-se um objeto de estudo relevante para o público acadêmico da área de \textit{software}, pois permite entender como diferentes modelos se comportam em termos de eficiência computacional, tempo de resposta, consumo de recursos e qualidade das soluções geradas. Compreender essas diferenças é essencial para orientar escolhas mais conscientes no uso de IA (Inteligência Artificial) como ferramenta de apoio tanto no aprendizado quanto na prática profissional.

O trabalho está organizado da seguinte forma: na Seção I, apresenta-se a introdução e a motivação do estudo. A Seção II fornece uma contextualização teórica sobre o tema. Em seguida, a Seção III descreve os principais conceitos de Inteligência Artificial necessários para a compreensão do trabalho. A Seção IV apresenta os modelos avaliados, enquanto a Seção V detalha a metodologia adotada nos experimentos. A Seção VI expõe e discute os resultados obtidos. Por fim, as Seções VII e VIII apresentam, respectivamente, a conclusão do estudo e possíveis direções para trabalhos futuros.
