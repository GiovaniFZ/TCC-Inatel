\section{Introdução}
Inteligência Artificial (IA) consolidou-se como uma das tecnologias mais transformadoras da era contemporânea, impactando áreas que vão desde a saúde e educação até a indústria e o setor financeiro. O rápido avanço dos modelos de IA, em especial dos modelos de aprendizado profundo (Deep Learning) e de linguagem natural (LLMs – Large Language Models), trouxeram benefícios significativos, como aumento da produtividade e automação de processos complexos. 
Nesse contexto, a eficiência dessas tecnologias tem sido amplamente destacada, visto que sua capacidade de processar grandes volumes de dados em tempo reduzido supera amplamente soluções tradicionais de computação.
Porém, conforme a Inteligência Artificial se torna cada vez mais presente no nosso dia a dia, crescem também as preocupações em relação à sua segurança. Se não forem bem planejados ou protegidos, os modelos de IA podem ser manipulados, apresentar resultados distorcidos ou até sofrer ataques que colocam em risco sua confiabilidade.
Diante desse cenário, a análise comparativa entre diferentes modelos de Inteligência Artificial não pode restringir-se apenas a métricas de desempenho computacional, como acurácia, tempo de resposta ou uso de recursos. É igualmente essencial compreender como esses modelos se comportam frente a ameaças de segurança, quais mecanismos de defesa possuem e até que ponto conseguem equilibrar a eficiência operacional com a robustez contra riscos.
O presente trabalho propõe, portanto, um estudo comparativo entre diferentes modelos de Inteligência Artificial, com ênfase em dois eixos principais: eficiência e segurança. A análise buscará inicialmente avaliar as capacidades técnicas dos modelos sob o aspecto de desempenho, para em seguida aprofundar-se nos desafios relacionados à segurança, discutindo suas vulnerabilidades.

\begin{comment}
O termo IA foi criado em 1956, por John McCarthy. Nessa década, os computadores possuíam hardware muito limitados comparados ao que temos hoje em dia. Mesmo assim, havia uma crença de que os computadores poderiam não apenas copiar, mas ir além do que o humano podia pensar.\\
A IA é uma área da ciência da computação que está relacionada com outras, como psicologia, a biologia, filosogia, etc. Dessa forma, é correto dizer que a IA consegue resolver vários tipos de problemas de diversas áreas diferentes. \\
\end{comment}

