\section{História da inteligência artificial}
O termo IA foi criado em 1956, por John McCarthy. Nessa década, os computadores possuíam hardware muito limitados comparados ao que temos Mesmo assim, havia uma crença de que os computadores poderiam não apenas copiar, mas ir além do que o humano podia pensar.
A partir da década de 1980, a Inteligência Artificial (IA) entrou em uma nova fase marcada pelo renascimento do interesse acadêmico e industria. Assim, houve o ressurgimento das redes neurais artificiais, impulsionado pelo algoritmo de retropropagação de erros (backpropagation). Em 1997, foi apresentada a arquitetura LSTM (Long Short-Term Memory), que ampliou a capacidade de aprendizado de sequências temporais e se tornou fundamental em tarefas de reconhecimento de fala e processamento de linguagem natural \cite{historiaIA}. No mesmo ano, o supercomputador Deep Blue, desenvolvido pela IBM, derrotou o campeão mundial de xadrez Garry Kasparov, simbolizando o poder da IA em domínios complexos e estratégicos \cite{IBM}.
O início do século XXI trouxe uma revolução impulsionada pelo crescimento do poder computacional e pela disponibilidade de grandes volumes de dados.Um marco decisivo ocorreu em 2017 com a introdução do Transformer, arquitetura que se tornou a base dos grandes modelos de linguagem (LLMs), possibilitando o surgimento de sistemas como o GPT-3 (2020), que revolucionou a forma de interação entre humanos e máquinas \cite{historiaIA}.Atualmente, a IA encontra-se em um novo auge com o crescimento da IA generativa, capaz de produzir textos, imagens e sons de maneira autônoma.

\begin{comment}
\subsection{Anos 90}
Os anos 90 foram muito importantes para a computação. Apesar de a ideia de termos um aparato mecânico que pensava como um humano ter surgido em 1726, no romance "As Viagens de Gulliver" de Jonathan Swift, foi apenas em 1914 que Leonardo Torres criou a primeira máquina que jogava xadrez com um humano. Com o tempo, as máquinas foram se aperfeiçoando, dessa forma, surgiram os primeiros computadores, (ex:ABC em 1939); os primeiros robôs (ex.:Shakey em 1966) e as primeiras redes neurais (ex.: SNARC em 1951)

\subsection{Anos 2000}
Nesse período, outros robôs e redes neurais foram surgindo cada vez mais, mas também houve melhorias no que tínhamos anteriormente, o que permitiu por exemplo com que a AlphaGo, um programa de inteligência artificial, vencesse Lee Sedol no jogo "Go" em 2016. 

\subsection{Mas, e o ChatGPT?}
O ChatGPT foi introduzido pela OpenAI em 2020. Ele ganhou notoriedade por conta de um diferencial: Ele apresentou como um dos primeiros LLMs, ou seja, um grande modelo de linguagem que tem o poder de responder a diversas perguntas e realizar tarefas mesmo não sendo treinado especificamente para uma coisa só.
\end{comment}