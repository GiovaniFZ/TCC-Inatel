\section{Conclusão}
A partir da análise comparativa entre o \textit{GPT-4.1-mini} e o \textit{Gemini 2.5 Flash}, foi possível compreender não apenas as diferenças técnicas entre os modelos, mas também como cada um deles pode apoiar de forma prática o processo de ensino de estudantes de desenvolvimento de \textit{software}. Para quem está em fase de aprendizagem, compreender código, corrigir erros e gerar soluções completas são etapas fundamentais. Dessa forma, os resultados mostram que ambos os modelos podem atuar como ferramentas auxiliares nessa área.

O \textit{GPT-4.1-mini} demonstrou ser especialmente útil para estudantes que precisam de respostas rápidas, diretas e fáceis de entender. Seu estilo mais objetivo favorece a assimilação inicial de conceitos, auxiliando na visualização de estruturas básicas do código, na identificação de erros simples e compreensão de lógicas fundamentais sem excesso de informação. Além disso, por ser mais barato e rápido, como mostrado nos resultados deste trabalho, torna-se uma opção acessível para uso frequente durante estudos diários, exercícios práticos e esclarecimento de dúvidas pontuais.

Por outro lado, o \textit{Gemini 2.5 Flash} mostrou-se mais adequado em situações em que o estudante deseja aprofundar o conhecimento, explorando detalhes mais avançados, boas práticas, tratamentos de erro completos e diferentes abordagens possíveis para um mesmo problema. Seu estilo mais detalhado permite que o programador enxergue a lógica por trás das soluções e compreenda o raciocínio necessário para construir aplicações completas.

No que diz respeito às métricas quantitativas, apesar de o GPT ter se destacado pelo custo significativamente menor, observa-se que, para estudantes, o valor adicional do Gemini não se torna inviável no uso cotidiano, especialmente considerando o nível de profundidade e detalhamento que o modelo oferece.

Portanto, conclui-se que não é possível afirmar que exista um único "melhor" modelo de LLM para todas as situações. Cada um apresenta pontos fortes e limitações que devem ser considerados de acordo com a necessidade do estudante: o \textit{GPT-4.1-mini} atende melhor demandas por respostas objetivas e rápidas, enquanto o \textit{Gemini 2.5 Flash} é mais indicado quando o objetivo é obter explicações aprofundadas e soluções mais completas. Sendo assim, a escolha ideal depende do contexto, do tipo de tarefa e do nível de conhecimento que o estudante busca desenvolver.