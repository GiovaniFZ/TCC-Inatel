\section{Trabalhos Futuros}

A partir dos resultados obtidos neste estudo, diversas linhas de continuidade podem ser exploradas em pesquisas futuras. Uma possibilidade relevante é ampliar o escopo de modelos avaliados, incluindo LLMs projetados especificamente para tarefas relacionadas ao desenvolvimento de \textit{software}. Entre esses modelos, destacam-se o \textit{Amazon Q Developer}, focado em suporte integrado a ambientes AWS (\textit{Amazon Web Services}), e o \textit{Claude Code}, uma variação do \textit{Anthropic Claude} otimizada para gerar, analisar e manipular código em múltiplas linguagens. Esses modelos apresentam propostas distintas, com forte ênfase em aplicações práticas e fluxo real de trabalho de engenheiros de \textit{software}, o que poderia enriquecer comparações de desempenho e efetividade.

Outra possível extensão consiste em incluir cenários de avaliação mais complexos, como resolução de problemas de arquitetura, otimização de algoritmos, testes automatizados e análise de grandes bases de código. Além disso, estudos podem incorporar métricas adicionais, como consumo energético e capacidade de manter coerência em interações prolongadas.