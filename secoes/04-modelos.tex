\section{Modelos de IA avaliados}
\subsection{ChatGPT}
Um dos exemplos mais conhecidos de LLM é a série Generative Pre-trained Transformer (GPT), desenvolvida pela OpenAI. Os modelos são pré-treinados em grandes quantidades de dados de texto por meio de aprendizado não supervisionado e aprendem a identificar relacionamentos e padrões nos dados de linguagem. Uma vez pré-treinados, os modelos podem ser ajustados para tarefas como resposta a perguntas, criação de conteúdo, sumarização de texto e geração de código de software. \\
No entanto, o ChatGPT também apresenta vários tipos de riscos e implicações. A capacidade do ChatGPT de gerar respostas convincentes pode ser explorada por atores mal-intencionados para disseminar desinformação, lançar ataques de phishing ou até mesmo se passar por indivíduos \cite{GPT-SecurityRisks}.

\subsection{Gemini}
Outro exemplo de uma LLM que vem ganhando força é o Gemini, do Google. Ele foi inicialmente desenvolvido em 2023 pela Google com o nome de Bard. A mudança do nome ocorreu em 2024, pois a empresa criou não apenas um produto de IA, mas sim uma família de modelos que foram se aperfeiçoando com o passar do tempo. Dessa forma, o produto não transmitiria a ideia de que ele seria "apenas um poeta", mas sim algo muito maior. \cite{CMSWire} \\
O Gemini funciona da seguinte forma: Primeiro, o modelo é pré-treinado, passa por um processo de pós-treino, gera as respostas aos prompts e, finalmente, o feedback humano é analisado. Porém, assim como todos os modelos de IA, o Gemini também possui algumas limitações, especialmente no que diz respeito às respostas, pois, algumas vezes, elas serão falsos positivos ou falsos negativos, além do mais, podem ocorrer vieses ou aluncinações. Dessa forma, é válido utilizá-las com cautela. \cite{Google} \\
Apesar de parecidos, há algumas diferenças no Gemini em relação a outros modelos de IA. Uma das vantagens por exemplo é a presença de 1 milhão de tokens \cite{Zapier}, com isso, temos uma capacidade maior de fazer mais perguntas e respostas, sem perder o contexto.