\section {A inteligência artificial e seus conceitos}
Para entender a pesquisa proposta, é importante ressaltar alguns conceitos iniciais:
\begin{enumerate}
    \item Inteligência Artifical (IA): Permite que computadores e máquinas consigam simular a aprendizagem como os humanos a partir de dados \cite{IBM-AI}.
    \item \textit{Machine Learning} (ML): Se refere a união de algoritmos que possuem a capacidade de aprender padrões de treinamento a partir de dados \cite{IBM-ML}.
    \item \textit{Deep Learning}: Também conhecido como aprendizado profundo, é uma área da \textit{Machine Learning} que utiliza várias redes neurais. Estas redes servem para simular a maneira como o cérebro humano funciona, em relação às tomadas de decisão. \cite{IBM-DL}
    \item IA Generativa: Tecnologia que permite com que modelos de IA consigam gerar conteúdos multimídia, como vídeos e imagens, como por exemplo, o \textit{Veo3} do Google para geração de vídeos \cite{googleVeo3}.
    \item Grandes Modelos de Linguagem -- LLMs (\textit{Large Language Models}): São modelos que possuem a capacidade de reconhecer imagens, textos e outros conteúdos e gerar predições a partir deles. O termo 'amplo' se dá por conta que estes modelos são treinados com muitos tipos de dados, e vários algoritmos de ML são utilizados para isso.
    \item \textit{Tokens}: Muito utilizados por LLMs, os \textit{tokens} são blocos de textos utilizados para realizar as predições. Por exemplo, quando escrevemos uma pergunta para um modelo LLM, o modelo gera os \textit{tokens} a partir da pergunta e consome para a resposta, portanto eles são gerados antes da resposta. \cite{NVIDIA}
    \item \textit{Prompts}: São instruções ou entradas fornecidas ao modelo de linguagem para orientar a geração de respostas. Um \textit{prompt} pode ser uma pergunta, uma descrição, um trecho de código ou qualquer texto que direcione o comportamento do modelo. A qualidade e a clareza do \textit{prompt} influenciam diretamente a precisão e a utilidade da resposta gerada pelo sistema.
    \item \textit{Benchmarking}: É definido como o processo de medição em relação a diferentes tipos de produtos, serviços ou organizações. Nesse contexto, é muito utilizado no contexto empresarial para medir como uma empresa se compara a outras do mesmo segmento \cite{WhatIsBenchmarking}.
    
\end{enumerate}

Em conjunto, esses conceitos fornecem a base teórica necessária para compreender o funcionamento dos modelos analisados neste estudo e estruturam o entendimento das métricas avaliativas apresentadas nas seções seguintes.