%Use esse arquivo para incluir novos pacotes

\usepackage[%usado para determinar medidas
top=1.78cm,
bottom=1.78cm,
left=1.65cm,
right=1.65cm,
headsep=0cm,
%showframe
]{geometry}
%\usepackage[justification=centering]{caption}
\usepackage{inatel}%carregar algumas estilizacoes do inatel
\usepackage{times}
\usepackage{enumitem}%redefinir espacos itemize
\usepackage{graphicx}
\usepackage{url,hyperref}
\usepackage[utf8]{inputenc}
\usepackage{csquotes}
\usepackage{float}%mais controle para manipular figuras
\usepackage{caption}%manipular legenda da figura e tabela
\usepackage{mathtools}%equacoes
\usepackage[hang,flushmargin]{footmisc} 
\usepackage{xcolor}
\usepackage{wrapfig} %usado para envolver figura com texto
%\usepackage[portuguese]{babel}
\usepackage{fancyhdr}%criacao do cabecalho
\usepackage{etoolbox}
\usepackage{adjustbox}%mais controle para ajustar tamanho da tabela
\usepackage{comment}%ambiente para comentario
\usepackage{relsize} %usado por comandos \mathlarger
%\usepackage{mathptmx}
\usepackage{array}

\usepackage{listings}
\usepackage{xcolor}
\usepackage[utf8]{inputenc}

\lstdefinelanguage{JavaScript}{
  keywords={break, case, catch, class, const, continue, debugger, default, delete,
    do, else, export, extends, finally, for, function, if, import, in, instanceof,
    let, new, return, super, switch, this, throw, try, typeof, var, void, while, with, yield},
  keywordstyle=\color{blue},
  ndkeywords={boolean, null, true, false},
  ndkeywordstyle=\color{gray},
  identifierstyle=\color{black},
  sensitive=false,
  comment=[l]{//},
  morecomment=[s]{/*}{*/},
  commentstyle=\color{teal},
  stringstyle=\color{orange},
  morestring=[b]',
  morestring=[b]"
}

\lstdefinelanguage{Python}{
  keywords={def, return, if, elif, else, for, while, break, continue,
    pass, import, from, as, class, try, except, raise, with, yield, lambda},
  keywordstyle=\color{blue},
  ndkeywords={},
  ndkeywordstyle=\color{purple},
  identifierstyle=\color{black},
  sensitive=false,
  comment=[l]\#,
  commentstyle=\color{gray}\ttfamily,
  stringstyle=\color{red}\ttfamily,
  morestring=[b]',
  morestring=[b]"
}

\lstset{
  basicstyle=\ttfamily\scriptsize,
  breaklines=true,
  frame=single,
  numbers=none,
  captionpos=b,
}
%Referencia bibliografica
\usepackage[
    style=numeric,
    sorting=none,
    maxbibnames=10]{biblatex}
\addbibresource{referencia.bib}

%Idioma. Use "english" para trabalhos em inglês
\usepackage[utf8]{inputenc}
\usepackage[T1]{fontenc}